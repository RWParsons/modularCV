\texttt{\href{https://github.com/ropensci/GLMMcosinor}{GLMMcosinor}} \hfill \textbf{\href{https://docs.ropensci.org/GLMMcosinor}{CRAN and rOpenSci} January 2024} \par
\begin{itemize}
    \item An R package to fit a cosinor model to rhythmic data using the glmmTMB framework. 
    \item Unlike existing cosinor modelling packages, allows fitting of GLMs and mixed-models.
\end{itemize}

\texttt{\href{https://github.com/healthpolicyanalysis/hpa.spatial}{hpa.spatial}} \hfill \textbf{\href{https://healthpolicyanalysis.github.io/hpa.spatial/}{pkg site}} \par
\begin{itemize}
    \item An R package for accessing and manipulating spatial data, focusing on the Australian (health) context.
\end{itemize}

\texttt{\href{https://github.com/ropensci/predictNMB}{predictNMB}} \hfill \textbf{\href{https://docs.ropensci.org/predictNMB/}{CRAN and rOpenSci} March 2023} \par
\begin{itemize}
    \item An R package that allows the user to perform simulations to estimate the cost-effectiveness of using a prediction model to assign a healthcare intervention.
    \item Can be used to determine whether or when a clinical prediction model or clinical decision support system may be worthwhile before development or implementation.
\end{itemize}

\texttt{\href{https://github.com/gentrywhite/DSSP}{DSSP}} \hfill \textbf{\href{https://cran.r-project.org/package=DSSP}{CRAN} June 2022} \par
\begin{itemize}
    \item An R package that allows users to fit Bayesian spatial models with direct sampling (\textit{fast}).
    \item Draws samples from the direct sampling spatial prior model which is 100-1000 times faster than MCMC.
\end{itemize}

\texttt{\href{https://github.com/RWParsons/simMetric}{simMetric}} \hfill \textbf{\href{https://cran.r-project.org/package=simMetric}{CRAN}  January 2022} \par
\begin{itemize}
    \item An R package that provides functions to calculate useful metrics (and their Monte Carlo standard errors) for the assessment of statistical methods in simulation studies.
    \item This allows for easy integration with other simulation study frameworks and the tidyverse-style workflow.
\end{itemize}

\texttt{\href{https://github.com/RWParsons/circacompare}{circacompare}} \hfill \textbf{\href{https://cran.r-project.org/package=circacompare}{CRAN} February 2021} \par
\begin{itemize}
    \item An R package that allows users to analyse circadian datasets using nonlinear regression models.
    \item Documented with a \href{https://cran.r-project.org/web/packages/circacompare/vignettes/circacompare-vignette.html}{vignette}; also available as a \href{https://rwparsons.shinyapps.io/circacompare/}{shiny app} and in \href{https://github.com/RWParsons/circacompare_py}{python}.
\end{itemize}